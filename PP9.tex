%!TEX encoding = UTF-8 Unicode
%==================================================
%      PREAMBOLO e DICHIARAZIONI INIZIALI
%==================================================
\documentclass[10pt,oneside,a4paper]{article}

\usepackage[utf8]{inputenc} 
\usepackage[italian]{babel}
\usepackage[T1]{fontenc}
\usepackage{siunitx} %Inserisce automaticamente i dati con le unità  di misura correttamente formattate del SI (utilizzo: \SI{0.82}{m^2}, in generale \SI{misura con il punto decimale}{unità  di misura})
\sisetup{output-decimal-marker = {.}, separate-uncertainty = true, input-uncertainty-signs = \pm, detect-weight=true, detect-family=true} %per usare SI con il punto decimale
\usepackage{listings} %Per citare codice informatico formattandolo correttamente
\usepackage{amsmath,amsthm,verbatim,amssymb,amsfonts,amscd,graphicx,mathtools}
\usepackage[makeroom]{cancel}
\newcommand{\abs}[1]{\left\lvert\,#1\,\right\rvert}
\usepackage{geometry}
\usepackage{epigraph}
\usepackage{booktabs}	%tabelle migliorate
\usepackage{tablefootnote}	%note a piè di pagina in tabella
\usepackage{threeparttable} %tabella con note a piè di tabella
\usepackage{caption}	%descrizione per figure
\usepackage{dblfnote}
\captionsetup{tableposition=top,figureposition=bottom,font=small} %setup descrizione
\usepackage{float}
\usepackage{esvect} %vettori
\usepackage{longtable} %tabelle lunghe
\usepackage[dvipsnames]{xcolor}
\definecolor{sepia}{HTML}{80002A}
\usepackage[colorlinks=true, citecolor=black, linkcolor=sepia, urlcolor=black]{hyperref}
\usepackage{mathrsfs}
\usepackage{circuitikz}
\tikzset{
  font={\fontsize{7pt}{12}\selectfont}}
\ctikzset{bipoles/resistor/height=0.2}
\ctikzset{bipoles/resistor/width=0.4}
\ctikzset{bipoles/diode/height=0.3}
\ctikzset{bipoles/diode/width=0.3}
\ctikzset{tripoles/american nand port/height=0.7}
\ctikzset{tripoles/american nand port/width=0.8}
\usepackage{enumitem} %Liste senza spazi verticali
\setlist{noitemsep}
\usepackage{amsmath}


\interfootnotelinepenalty=10000


\usepackage{multicol}
\newenvironment{Figure}
  {\par\medskip\noindent\minipage{\linewidth}}
  {\endminipage\par\medskip}

\newcommand{\var}{\operatorname{var}}
\newcommand{\cov}{\operatorname{cov}}


\usepackage{listings} %Per inserire codice
\lstdefinestyle{CStyle}{
    backgroundcolor=\color{backgroundColour},   
    commentstyle=\color{mGreen},
    keywordstyle=\color{magenta},
    numberstyle=\tiny\color{mGray},
    stringstyle=\color{mPurple},
    basicstyle=\footnotesize\ttfamily,
    breakatwhitespace=false,         
    breaklines=true,                 
    captionpos=b,                    
    keepspaces=true,                 
    numbers=left,                    
    numbersep=5pt,                  
    showspaces=false,                
    showstringspaces=false,
    showtabs=false,                  
    tabsize=2,
    language=C
}

\definecolor{color1}{RGB}{90,0,0} % Color of the article title and sections
\definecolor{color2}{RGB}{0,20,50} % Color of the boxes behind the abstract and headings
\definecolor{mGreen}{rgb}{0,0.6,0}
\definecolor{mGray}{rgb}{0.5,0.5,0.5}
\definecolor{mPurple}{rgb}{0.58,0,0.82}
\definecolor{backgroundColour}{rgb}{0.95,0.95,0.92}


%==================================================
%                  PRIMA PAGINA
%==================================================

\title{\textsc{\textbf{Esercitazione 9}: DFT con Arduino}}
\author{\small{G. Galbato Muscio} \and \small{L. Gravina} \and \small{L. Graziotto}}
\date{18 Dicembre 2018}

\begin{document}
	\begin{figure}
		\centering
		\includegraphics[scale=0.5, trim={2.8cm 8.9cm 0 9cm}, clip]{logo.png}
	\end{figure}
	\maketitle
	\begin{center} 
		\fbox{{\fontsize{12pt}{8mm}\textsc{Gruppo 11}}} \\
	\end{center}
\hrule
\vspace{0.5cm}
\renewcommand{\abstractname}{Abstract}
\begin{abstract}
Si utilizza il microcontrollore Arduino come hardware per il campionamento di segnali analogici e il software Processing per la loro analisi in frequenza. Si studiano segnali periodici e segnali impulsivi in presenza e assenza di rumore.
\end{abstract}
\vspace{4cm}
\tableofcontents %Indice
\newpage


\pagebreak


\begin{multicols}{2}
%==================================================
%             APPARATO STRUMENTALE
%==================================================
\section{Apparato strumentale}
Si vuole campionare il segnale analogico in ingresso utilizzando la più alta frequenza possibile: le misure vengono dunque memorizzate da Arduino in un buffer nella memoria e solo alla fine vengono trasmesse in seriale, in tal modo si riduce il tempo per effettuare la singola misura\footnote{La trasmissione seriale impiega un tempo non trascurabile per essere completata.}, riducendolo approssimativamente a quello necessario per completare la funzione \texttt{analogRead()}, pari a circa $t_\text{A} = \SI{111}{\micro\second}$, la massima frequenza di campionamento è dunque $f_\text{max} = t_\text{A}^{-1} = \SI{9.0}{\kilo\hertz}$. Considerando la memoria dell'Arduino possono essere memorizzati fino a $N = 800$ valori, dunque campionando alla frequenza massima la durata del campionamento è di circa $t_\text{camp} = N/f_\text{max} = \SI{8.89}{\milli\second}$.

Per evitare fenomeni di \emph{aliasing}, deve essere verificata la seguente condizione sulla frequenza del segnale in ingresso
\begin{equation}\label{eq:Nyquist}
	f_\text{in} \leq \frac{f_\text{max}}{2} = \SI{4.5}{\kilo\hertz} \equiv f_\text{Nyquist}.
\end{equation}


%==================================================
%             SEGNALE ANALOGICO SENZA RUMORE
%==================================================
\section{Segnale analogico in assenza di rumore}
Si campionano e si studiano in frequenza dei segnali analogici prodotti dal generatore di funzioni in assenza di rumore artificiale.
\subsection{Segnale sinusoidale}
Si campiona digitalmente un segnale sinusoidale analogico in ingresso a frequenze $f_1 = \SI{111 \pm 111}{\hertz}, \; f_2 = \SI{111 \pm 111}{\kilo\hertz}$ e infine $f_3 = \SI{111 \pm 111}{\kilo\hertz}$, quest'ultima volutamente superiore al limite dettato dalla frequenza di Nyquist (\ref{eq:Nyquist}) per poter apprezzare i fenomeni di \emph{aliasing}; le frequenze sono misurate con un oscilloscopio.

Si riportano in Figura \ref{fig:sin200camp} e in Figura \ref{fig:sin200trasf} il segnale a frequenza $f_1$ campionato da Arduino e la sua trasformata di Fourier discreta, rispettivamente. Sull'asse delle ascisse sono riportate le frequenze di Fourier tenendo presente la relazione che lega la frequenza con l'indice della trasformata
\begin{equation}\label{eq:frequenzeFourier}
	f_\text{fourier} = k \frac{f_\text{s}}{N}
\end{equation}
essendo $f_\text{s}/N$ l'inverso della finestra temporale del campionamento, e dunque la minima frequenza misurabile: le altre frequenze della decomposizione saranno multipli interi della frequenza elementare.
Nel grafico della trasformata viene inoltre riportata indicativamente la frequenza di Nyquist.

%\begin{Figure}
%	\begin{center}
%	\includegraphics[width=\linewidth]{sin200camp.pdf}
%	\captionof{figure}{Segnale sinusoidale analogico a frequenza $f=\SI{111 \pm 111}{\hertz}, campionato con Arduino.}
%	\label{fig:sin200camp}
%	\end{center}
%\end{Figure}
%
%\begin{Figure}
%	\begin{center}
%	\includegraphics[width=\linewidth]{sin200trasf.pdf}
%	\captionof{figure}{Trasformata di Fourier discreta di un segnale sinusoidale analogico a frequenza $f=\SI{111 \pm 111}{\hertz}.}
%	\label{fig:sin200trasf}
%	\end{center}
%\end{Figure}

Come da aspettativa teorica, la trasformata ha un picco in corrispondenza della frequenza del segnale in ingresso, essendo questo sinusoidale ed essendo lontani dalla frequenza di Nyquist.

Analogamente, si riportano in Figura \ref{fig:sin42camp}, \ref{fig:sin42trasf}, \ref{fig:sin6camp} e \ref{fig:sin6trasf} i grafici ottenuti campionando due segnali di frequenza $f_2$ e $f_3$. L'ultimo è a frequenza superiore a quella di Nyquist, la frequenza apparente è coerente con quanto previsto teoricamente, ovvero
\begin{equation}\label{eq:frequenzaAlias}
	f_\text{alias} = 2f_\text{Nyquist} - f_\text{in}.
\end{equation}

%\begin{Figure}
%	\begin{center}
%	\includegraphics[width=\linewidth]{sin42camp.pdf}
%	\captionof{figure}{Segnale sinusoidale analogico a frequenza $f=\SI{111 \pm 111}{\hertz}, campionato con Arduino.}
%	\label{fig:sin42camp}
%	\end{center}
%\end{Figure}
%
%\begin{Figure}
%	\begin{center}
%	\includegraphics[width=\linewidth]{sin42rasf.pdf}
%	\captionof{figure}{Trasformata di Fourier discreta di un segnale sinusoidale analogico a frequenza $f=\SI{111 \pm 111}{\hertz}.}
%	\label{fig:sin42trasf}
%	\end{center}
%\end{Figure}
%
%\begin{Figure}
%	\begin{center}
%	\includegraphics[width=\linewidth]{sin6camp.pdf}
%	\captionof{figure}{Segnale sinusoidale analogico a frequenza $f=\SI{111 \pm 111}{\hertz}, campionato con Arduino.}
%	\label{fig:sin6camp}
%	\end{center}
%\end{Figure}
%
%\begin{Figure}
%	\begin{center}
%	\includegraphics[width=\linewidth]{sin6trasf.pdf}
%	\captionof{figure}{Trasformata di Fourier discreta di un segnale sinusoidale analogico a frequenza $f=\SI{111 \pm 111}{\hertz}.}
%	\label{fig:sin6trasf}
%	\end{center}
%\end{Figure}

\subsection{Onda quadra}

%==================================================
%             STUDIO DEL RUMORE
%==================================================
\section{Studio del rumore}
\begin{center}
\begin{circuitikz}
\draw (0,0) node[op amp] (opamp) {}
(opamp.+) to[R=$R_1'$] ++(-2,0) node[circ] {} node[left] {$v_\text{2}$}
(opamp.-) to[short] ++(0, 1.25) to[R = $R_2$] ++(2,0) -| (opamp.out)
(opamp.-) node[circ] {}
(opamp.+) node[circ] {}
(opamp.-) to[R=$R_1$] ++(-2,0) node[circ] {} node[left] {$v_1$}
(opamp.out) to[short, *-*] ++(0.5,0) node[right] {$v_\text{o}$}
(opamp.+) to[R=$R_2'$] ++(0,-2) node[ground] {}
;\end{circuitikz}
\end{center}

\begin{center}
\begin{circuitikz}
\draw (0,0) node[op amp] (opamp) {}
(opamp.out) to[short, *-*] ++(0.5,0) node[right] {$V_\text{o}$}
(opamp.-) to[short] ++(-0.25,0) to[R=$R$] ++(-1.5,0) node[ground] {}
(opamp.-) -| ++(0,1) to[R = $R'$] ++(2,0) -| (opamp.out)
(opamp.-) node[circ] {}
(opamp.+) to[R=$R_A$] ++(-2,0) node[circ] {} node[left]{$V_A$}
(opamp.+) node[circ] {}
(opamp.+) to[short] ++(0,-1) to[R=$R_B$] ++(-2,0) node[circ] {} node[left]{$V_B$}
;\end{circuitikz}
\end{center}


%==================================================
%             SEGNALE IMPULSIVO
%==================================================
\section{Studio di un segnale impulsivo}
\subsection{Impulso in assenza di rumore}
\subsection{Impulso in presenza di rumore}


\end{multicols}
\newpage
\section{Appendice}


%ESEMPIO DI CODICE
%\begin{lstlisting}[style=CStyle, caption={Codice per la misura dei tempi di esecuzione della moltiplicazione}, label=lst:moltiplicazione]
%#define MAX 10000
%
%int p, q;
%long unsigned t0, t1;
%
%void setup() {
%	Serial.begin(9600);
%	p=10; 
%}
%
%void loop() { 
%	Serial.print(p);
%	t0 = micros();
%	q = p*10;
%	t1 = micros();
%	p = p+1;	
%	Serial.print(p);
%	Serial.print(" ");
%	Serial.println(t1-t0); //stampa il tempo
%	while (p > MAX) { }; //il programma si arresta quando il numero supera MAX
%	}
%\end{lstlisting}


%ESEMPIO DI FIGURA
%\begin{Figure}
%	\begin{center}
%	\includegraphics[width=\linewidth]{comune.png}
%	\captionof{figure}{Istantanea dell'oscilloscopio per l'amplificatore differenziale, misura di $A_c$}
%	\label{fig:Ac_differenziale}
%	\end{center}
%\end{Figure}


%ESEMPIO DI TABELLA
%\begin{center}
%\captionof{table}{Misure per la stima di $A_c$}
%\label{tab:Ac_differenziale}
%\begin{tabular}{c|c|c|c}
%$f$ [\SI{}{Hz}] & $V_i$ [\SI{}{V}] & $v_o$ [\SI{}{mV}] & $A_c = v_o / V_i$ \\
%\hline
%      149.5 &        3.90 &         11.3 & 2.90e-03 \\
%      222.0 &        3.90 &         11.5 & 2.95e-03 \\
%      281.0 &        3.90 &         11.8 & 3.03e-03 \\
%      359.0 &        3.90 &         11.8 & 3.03e-03 \\
%      461.0 &        3.90 &         12.2 & 3.13e-03 \\
%\hline
%\end{tabular}
%\end{center}



\end{document}